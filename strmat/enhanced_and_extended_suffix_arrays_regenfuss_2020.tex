\documentclass[a4paper,10pt]{article}

\usepackage[utf8]{inputenc}
\usepackage[english]{babel}
\usepackage{natbib}

\title{Enhanced and Extended Suffix Arrays}
\author{Adrian Regenfuß}

\begin{document}

\maketitle

\begin{abstract}
In this report, I review the literature on enhanced and extended suffix arrays
in the context of searching long strings. I examine the different algorithms
used for both constructing enhanced and extended suffix arrays and for using
them in searching long strings.
In the end, I compare enhanced and extended suffix arrays with suffix arrays
and suffix trees.
\end{abstract}

\section*{Introduction}

Finding the occurences of one string in another string, longest repeated
substrings and longest shared substrings of two different strings are
fundamental problems for many kinds of computing systems.

As a result, many different algorithms have been developed for these
kinds of problems: For finding the occurences of one string in another one
the naive algorithm and the Boyer-Moore algorithm, %TODO: citation needed
and for all three of these problems (and more) three different data structures:
the suffix tree, %TODO: citation needed
the suffix array \citealt{manber1993suffix} and the enhanced suffix array. %TODO: citation needed (Manber and Myers 1990 and Abouelhoda, Kurtz and Ohlebusch 2002

Suffix trees, suffix arrays and enhanced suffix arrays have the
disadvantage of requiring to be constructed for a specific string,
which has time and space requirements.
Because of this, they are better suited for tasks where unchanging %TODO: is unchanging the right word here?
strings have to
be searched or matched repeatedly, although there has been some work to
extend the suffix array to dynamic strings. %TODO: Salson et. al 2009

Since searching and matching very long unchanging strings is very common
in genome analysis, it doesn't surprise that both suffix arrays and
enhanced suffix arrays were developed in that context.

\section*{Different String Matching Problems}

A plethora of different string matching problems have been identified
by computer scientists, for many of which suffix arrays and enhanced
suffix arrays are useful.

\subsection*{Searching}

%Problems:
%Repeats:
% • Maximal Repeats
% • Supermaximal repeats
%Searching
% • Longest common substring

\section*{Suffix Trees}

\section*{Suffix Arrays}

\subsection*{Construction}

\subsection*{Searching}

\section*{Enhanced and Extended Suffix Arrays}

Enhanced suffix arrays were first proposed in Abouelhoda et al. 2004 as an
improvement over normal suffix arrays. An enhanced suffix array contains
a suffix array together with the LCP-array of the string, and sometimes
a Burrows-Wheeler transformation and an inverse of the suffix table.

\subsection*{suftab}

\subsection*{lcptab}

\subsection*{bwttab}

\subsection*{suftab$^{-1}$}

%Why do they think the lcp-tree is so important?
%Tree bottom-up traversal is from the root to the leaves,
%top-down is the other way around

\subsection*{LCP-Interval Trees}

\subsection*{Searching}

\subsection*{Finding Maximal and Supermaximal Repeats}

\section*{Comparison}

\section*{Applications}

\section*{Conclusion}

\bibliographystyle{plainnat}
\bibliography{sources}

\end{document}
