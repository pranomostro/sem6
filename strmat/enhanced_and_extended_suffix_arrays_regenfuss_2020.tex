\documentclass[a4paper,10pt]{article}

\usepackage[utf8]{inputenc}
\usepackage[english]{babel}
\usepackage{natbib}

\title{Enhanced and Extended Suffix Arrays}
\author{Adrian Regenfuß}

\begin{document}

\maketitle

\begin{abstract}
In this report, I review the literature on enhanced and extended suffix arrays
in the context of searching long strings. I examine the different algorithms
used for both constructing enhanced and extended suffix arrays and for using
them in searching long strings.
In the end, I compare enhanced and extended suffix arrays with suffix arrays
and suffix trees.
\end{abstract}

\section*{Introduction}

Finding the occurences of one string in another string, longest repeated
substrings and longest shared substrings of two different strings are
fundamental problems for many kinds of computing systems.

As a result, many different algorithms have been developed for these kinds
of problems: For finding the occurences of one string in another one the
naive algorithm and the Boyer-Moore algorithm (\citealt{boyer1977fast}),
and for all three of these problems (and more) three different data
structures: the suffix tree (\citealt{weiner1973linear}), the suffix
array (\citealt{manber1993suffix}) and the enhanced suffix array
(\citealt{abouelhoda2002enhanced}).

Suffix trees, suffix arrays and enhanced suffix arrays have the
disadvantage of requiring to be constructed for a specific string,
which has time and space requirements.  Because of this, they are better
suited for tasks where immutable strings have to be searched or matched
repeatedly, although there has been some work to extend the suffix array
to dynamic strings (\citealt{salson2010dynamic}).

Since searching and matching very long immutable strings is very common
in genome analysis, it doesn't surprise that both suffix arrays and
enhanced suffix arrays were developed in that context.

This report first describes the enhanced suffix array as a data structure,
then sketches the algorithms used for constructing it, and afterwards
describes different string matching problems and how they are solved
by enhanced and extended suffix arrays. Finally, it compares enhanced
suffix arrays to normal suffix arrays and suffix trees, and closes with
an overview of tools that implement enhanced and extended suffix arrays.

\section*{Enhanced and Extended Suffix Arrays}

Enhanced suffix arrays were first proposed in
\citealt{abouelhoda2002enhanced} as an improvement over normal suffix
arrays. An enhanced suffix array contains a suffix array together with the
LCP-array of the string, and sometimes a Burrows-Wheeler transformation
and an inverse of the suffix table.

For the following, let $S$ be a finite string of length $n$ over the
finite alphabet $\Sigma$.

\subsection*{suftab}

The suffix array $\hbox{suftab}$ is an array of integers describing the
positions of of sorted suffixes of $S$ in $S$.

More formally, let $\hbox{Suf}_{S}$ be the set of suffixes of $S$. Let
then $\hbox{SortSuf}_{S}$ be the array of lexically sorted suffixes of
$S$. Then $\hbox{suftab}[i]=k$ if and only if $S[k..n]=\hbox{SortSuf}[i]$.

\subsection*{lcptab}

The LCP (longest common prefix) table describes the length of the longest
common prefix of two neighbouring entries in the array of sorted suffixes.

Formally, $\hbox{lcptab}[i]=k$ iff
$\hbox{SortSuf}[i][0..k]=\hbox{SortSuf}[i-1][0..k]$. The zeroth entry
in $\hbox{lcptab}$ is always $0$.

\subsection*{bwttab}

Informally, $\hbox{bwttab}$ contains the character before the suffix
in $\hbox{suftab}$.

\subsection*{suftab$^{-1}$}

\subsection*{Example}

For example, let $S=\hbox{"cagccacat"}$.  Then $\hbox{suftab}$,
$\hbox{lcptab}$, $\hbox{bwttab}$, $\hbox{suftab}^{-1}$ and
$\hbox{suftab}$ are the following:

%TODO: add inverse suffix array

\begin{center}
	\begin{tabular}{ | l | c | c | c | l | }
		\hline
		i & suftab & lcptab & bwttab & SortSuf \\ \hline
		0 & 5 & 0 & c & acat\$ \\ \hline
		1 & 1 & 1 & c & agccacat\$ \\ \hline
		2 & 7 & 1 & c & at\$ \\ \hline
		3 & 4 & 0 & c & cacat\$ \\ \hline
		4 & 0 & 2 &  & cagccacat\$ \\ \hline
		5 & 6 & 2 & a & cat\$ \\ \hline
		6 & 3 & 1 & g & ccacat\$ \\ \hline
		7 & 2 & 0 & a & gccacat\$ \\ \hline
		8 & 8 & 0 & a & t\$ \\ \hline
		9 & 9 & 0 & t & \$ \\ \hline
	\end{tabular}
\end{center}

\subsection*{LCP-Interval Trees}

%Tree bottom-up traversal is from the root to the leaves,
%top-down is the other way around

\section*{Different String Matching Problems}

A plethora of different string matching problems have been identified
by computer scientists, for many of which suffix arrays and enhanced
suffix arrays are useful.

\citealt[pg. 2]{abouelhoda2004replacing} summarizes suffix tree applications
from \citealt[chap. 2]{gusfield1997algorithms} and classify them after their
type of tree traversal.

\subsection*{Exact String Matching}

\subsubsection*{Description}

Give a string $S$ of length $n$ and a string $T$ of length $m$ with $m
\le n$ both using them same alphabet $\Sigma$, the exact matches of $T$
in $S$ is the set of indices $I=\{i_1, \ldots, i_k\}$ for which holds
that $\forall i \in I: S[i..i+m]=T$.

In other words, $I$ is the set of indices where $T$ is a substring in $S$.

\subsubsection*{Suffix Array Algorithm}

\citealt{manber1993suffix} describes an exact string matching algorithm
that runs in ${\cal{O}}(m \log|\Sigma)$ time and constant space.

Their proposed algorithm uses binary search to sequentially find
the first (smallest) index $L_{W}$ and the last (biggest) index
$R_{W}$ in $\hbox{suftab}$ so that $S[\hbox{suftab}[L_{W}]]$ and
$S[\hbox{suftab}[R_{W}]]$ have the prefix $T$.

First, it searches $L_{W}$ using binary search on the whole array
$\hbox{suftab}$ and then uses $L_{W}$ as a left boundary to find $R_{W}$,
which often improves runtime as opposed to two independent searches
over the whole array, although the latter might be easier to parallelize.

\subsubsection*{Extended Suffix Array Algorithm}

\citealt{manber1993suffix} then propose a speed improvement based on
longest common prefixes. Their method attempts to reduce the number of
single-character comparisons by only comparing characters that occur
after the longest common prefix of $T$ and $S[M_{W}]$ ($M_{W}$ being
the index in the middle between $L_{W}$ and $R_{W}$).

%They then go on to explain a more complicated method,
%and Gusfield expands on their work on pg. 152 (84 in the PDF).

I have not come across a proposal to use interpolation search first
described in \citealt{perl1978interpolation} to search the suffix array
with an improved ${\cal{O}}(\log \log n)$ runtime.  This perhaps stems
from the fact that interpolation search assumes uniform distribution of
the alphabet, and has a worst-case runtime of ${\cal{O}}(n)$ runtime. It
still might be useful to empirically test speed differences in binary
and interpolation search.

\subsection*{Supermaximal Repeats}

\subsubsection*{Description}

\subsubsection*{Enhanced Suffix Array Algorithm}

%Problems:
%Repeats:
% • Maximal Repeats
% • Supermaximal repeats
% • Maximum Unique Matches
%Searching
% • Longest common substring

\section*{Construction}

\section*{Comparison}

\section*{Applications}

%REPuter
%vmatch
%MUMmer

\section*{Conclusion}

\bibliographystyle{plainnat}
\bibliography{sources}

\end{document}
